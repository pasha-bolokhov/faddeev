\sbibitem{1956}
{\LD} {Единственность решения обратной задачи
рассеяния} {\Vestnik} {11} {no.~7} {126--130} {1956}
% {Uniqueness of solution of the inverse scattering problem}

\sbibitem{1957b}
{\LD}
{О разложении произвольных функций по собственным функциям оператора Шредингера}
{\Vestnik} {12}
  {no.~7} {164--172} {1957}
% {On expansion of arbitrary functions in eigenfunctions of the
%  {S}chr\"odinger operator}

\sbibitem{1957a}
{\LD}
{О выражении для следа разности двух сингулярных
дифференциальных операторов типа Штурма--Лиувилля}
{\DAN} {115} {no.~5} {878--881} {1957}
% {An expression for the trace of the difference between two singular
%  differential operators of the {S}turm--{L}iouville type}

\rbibitem{1959b}
{\LD}
{О дисперсионных соотношениях в нерелятивистской теории рассеяния}
{\ZETF} {35} {no.~2} {433--439} {1958}

\ebibitem{1959b}
{\LF}
{Dispersion relations in non--relativistic scattering theory}
{\JETP} {8} {299--303} {1959}

\sbibitem{1958a} {О.А. Ладыженская и \LD} {К теории возмущений непрерывного
спектра} {\DAN} {120} {no.~6} {1187--1190} {1958}
% {O.A. Ladyzhenskaya and \LF}
% {On continuous spectrum perturbation theory}

\rbibitem{1958b}
{\LD}
{О связи S--матрицы и потенциала для одномерного оператора Шредингера}
{\DAN} {121} {no.~1} {63--66} {1958}

\ebibitem{1958b}
{\LF}
{The relation between $S$-matrix and potential for the
  one--dimensional Schr\"odinger operator}
{\SPD} {3} {747--751} {1959}

\rbibitem{1959a}
{\LD}
{Обратная задача квантовой теории рассеяния}
{\UMN} {14} {no.~4} {57--119} {1959}

\ebibitem{1959a}
{\LF}
{The inverse problem in the quantum theory of scattering}
{J. Math. Phys.} {4} {72--104} {1963}

\bibitemr{1958d}
{\LD} \\
{Свойства S--матрицы для рассеяния на локальном потенциале.}\\
{Автореферат дисс.\ на соиск.\ учен.\ степени канд.\ физ.--мат. наук}
{Л.: ЛГУ} {} {1959} {} {6~с} {1}

\rbibitem{1960b}
{\LD}
{Теория рассеяния для системы из трех частиц}
{\ZETF} {39} {no.~5} {1459--1467} {1960}

\ebibitem{1960b}
{\LF}
{Scattering theory for a three--particle system}
{\JETP} {12} {1014--1019} {1960}

\rbibitem{1960a}
{В.С. Буслаев и \LD}
{О формулах следов для
дифференциального сингулярного оператора типа Штурма--Лиувилля}
{\DAN} {132} {no.~1} {13--16} {1960}

\ebibitem{1960a}
{V.S. Buslaev and \LF} {Formulas for traces for a singular
Sturm--Liouville differential operator}
{\SMD} {1}  {451--454} {1960}

\rbibitem{1962a}
{Р.А. Минлос и \LD}
{Замечание о задаче трех частиц с точечным взаимодействием}
{\ZETF} {41} {no.~6} {1850--1851} {1961}

\ebibitem{1962a}
{R.A. Minlos and \LF}
{Comment on the problem of three particles with point interactions}
{\JETP} {14}  {1315--1316} {1962}

\rbibitem{1961b}
{Ф.А. Березин и \LD}
{Замечание об уравнении Шредингера с сингулярным потенциалом}
{\DAN} {137} {no.~5} {1011--1014} {1961}

\ebibitem{1961b}
{F.A. Berezin and \LF}
{Remark on the Schr\"odinger equation with singular potential}
{\SMD} {2} {372--375} {1961}

\rbibitem{1961a}
{\LD}
{Строение резольвенты оператора Шредингера системы трех
 частиц с парным взаимодействием}
{\DAN} {138} {no.~3} {565--567} {1961}

\ebibitem{1961a}
{\LF}
{The resolvent of the Schr\"odinger operator for a system of three
  particles interacting in pairs}
{\SPD} {6} {384--386} {1961}

\rbibitem{1962b}
{Р.А. Минлос и \LD}
{О точечном взаимодействии для системы из трех частиц в квантовой механике}
{\DAN} {141} {no.~6} {1335--1338} {1961}

\ebibitem{1962b}
{R.A. Minlos and \LF}
{On the point interaction for a three--particle system in quantum
  mechanics}
{\SPD} {6} {1072--1074} {1962}

\rbibitem{1963c}
{\LD}
{Строение резольвенты оператора Шредингера системы трех частиц и
задача рассеяния}
{\DAN} {145} {no.~2} {301--304} {1962}

\ebibitem{1963c}
{\LF}
{The construction of the resolvent of the Schr\"odinger operator for
  a three--particle system, and the scattering problem}
{\SPD} {7}  {600--602} {1963}

\sbibitem{1963a}
{\LD}
{Математические вопросы квантовой теории рассеяния для системы трех частиц}
{\Trudy} {69} {} {1--122} {1963}

[For English translation see the book \cite{b1963a}] 

\bibitemr{1963e}
{\LD,}\\
{Математические вопросы квантовой теории рассеяния для системы трех частиц.}
{Автореферат дисс.\ на соиск.\ учен.\ степени докт.\ физ.--мат. наук}
{М.: МИАН СССР} {} {1963} {} {16~с} {1}

\rbibitem{1963b}
{\LD}
{О разделении эффектов самодействия и рассеяния по теории возмущений}
{\DAN} {152} {no.~3} {573--576} {1963}

\ebibitem{1963b}
{\LF}
{On the separation of self--action and scattering effects in
  perturbation theory}
{\SPD} {8} {881--883} {1964}

\rbibitem{1964b}
{\LD}
{О модели Фридрихса в теории возмущений непрерывного спектра}
{\Trudy} {73} {} {292--313} {1964}

\ebibitem{1964b}
{\LF}
{On a model of Friedrichs in the theory of perturbations of the
  continuous spectrum}
{Am. Math. Soc., Transl., II. Ser.} {62} {177--203} {1967}

\rbibitem{1964c}
{\LD}
{Свойства S--матрицы одномерного уравнения Шредингера}
{\Trudy} {73} {} {314--336} {1964}

\ebibitem{1964c}
{\LF}
{Properties of the S--matrix of the one--dimensional {S}chr\"odinger
  equation}
{Am. Math. Soc., Transl., II. Ser.} {65} {139--166} {1967}

\rbibitem{1965b}
{В.Н. Попов и \LD}
{Об одном подходе к теории бозе--газа при низких температурах}
{\ZETF} {47} {no.~4} {1315--1321} {1964}

\ebibitem{1965b}
{V.N. Popov and \LF}
{An approach to the theory of the low--temperature {B}ose gas}
{\JETP} {20}  {890--893} {1965}

\bibitemr{1964aa}
{\LD}\\
{Операторы квантовой механики.} 
{\em Функциональный анализ. (Справочная мат.\ б--ка)}
{(М.: Наука} {} {1964} {} {279--322} {3}

%\bibitem{1964a}
%? В.П. Маслов и
%\LD,}\\
%{Операторы квантовой механики.} \\

\brbibitem{1964aa}
В кн.: {\em Функциональный анализ}. (Справочная мат.\ б--ка)
% Под ред.\ С.Г. Крейна
(М.: Наука, 1972), {423--454}.
(Изд.\ 2--е, перераб.\ и доп.) \\ [1mm]
% {Funktsionalnyi analiz} {Izdat. ``Nauka'', Moscow. 1972, 544~pp.
% Edited by S. G. Kre\u\i n, Second edition, revised and augmented,
%  Mathematical Reference Library.}\\

\bibiteme{1964aa}
{\LD}\\
{Операторы квантовой механики.} 
{\em Functional analysis} {Wolters--Noordhoff Publ.} {Groningen} {1972}
{2nd edition} {} {3}

\bibitemr{1964d}
{Ф.А. Березин, Р.А. Минлос и \LD} 
{Некоторые математические вопросы квантовой механики
систем с большим числом степеней свободы.} 
{\em Труды IV Всесоюзн.\ матем.\ съезда}
{Л.: Наука} {} {1964} {т.~2} {532-541} {2}

\rbibitem{1965c}
{\LD}
{Растущие решения уравнения Шредингера}
{\DAN} {165} {no.~3} {514--517} {1965}

\ebibitem{1965c}
{\LF}
{Increasing solutions of the Schr\"odinger equation}
{\SPD} {10} {1033--1035} {1966}

\rbibitem{1966}
{\LD}
{Факторизация S--матрицы многомерного оператора Шредингера}
{\DAN} {167} {no.~1} {69--72} {1966}

\ebibitem{1966}
{\LF}
{Factorization of the S--matrix for the multidimensional
  {S}chr\"odinger operator}
{\SPD} {11} {209--211} {1966}

\rbibitem{1967a}
{\LD}
{Разложение по собственным функциям оператора Лапласа на
фундаментальной области дискретной группы на плоскости Лобачевского}
{Труды Моск.\ мат. общества} {17} {} {323--350} {1967}
%{Trudy Moskov. Mat. Obshestva} {17} {} {323--350} {1967}

\ebibitem{1967a}
{\LF}
{Expansion in eigenfunctions of the Laplace operator on the
fundamental domain of a discrete group on the Lobachevski plane}
{Transl.\ Moscow Math.\ Soc.} {17} {357--386} {1969}

\bibitemr{1967b}
{В.Н. Попов и \LD} 
{Теория возмущений для калибровочно--инвариантных  полей.}
{Препринт ИТФ--67--036} {} {Киев} {1967} {} {28~с} {1}
%Preprint ITF--67--036 (Kiev, 1967), 28~pp. (in Russian)\\

\bibiteme{1967b}
{V.~N. Popov and L.D. Faddeev} 
{Perturbation theory for gauge--invariant fields.} 
{Preprint NAL--THY--57} {National Acceleration Laboratory} {} {1972} {} {36p.} {1}

\bbibitem{1967b}
Reprinted in: {\em Gauge theory of weak and electromagnetic
interactions} (ed. by C.H.Lai) (1980), 213--233. \\
Reprinted in: \fy{31--51} \\
Reprinted in: {\em 50 years of Yang--Mills theory}
(World Scientific, 2005) 39--64.

\bibitemr{1967e}
{\LD}
{О поле Янга--Миллса.} 
{\em Физика высоких энергий и теория элементарных частиц}
{Труды Междунар.\ школы по теор.\ физике, Ялта, 1966}
{Наукова думка} {Kiev} {1967} {} {766-769} {2}

\bibitemr{1967f}
{\LD} 
{Интегральные уравнения теории рассеяния для системы N частиц.} 
{\em Труды проблемного симпозиума по физике ядра, Тбилиси, 1967}
{} {М.} {1967} {т.~1} {43--56} {2}

\bibiteme{1967g}
{\LF,} 
{Integral equations for the three--particle scattering problem.}
{\em The physics of electronic and atomic collisions}
{} {Leningrad} {1967} {} {145--149} {3}

\sbibitem{1967h}
{\LF\ and V.N. Popov}
{Feynman diagrams for the Yang--Mills field}
{\PL} {B25} {} {29--30} {1967} \\

\bibitemr{1968c}
{\LD}
{Гамильтонова формулировка теории тяготения.}
В кн.: {\em Тезисы докладов 5--й Междунар.\ конф.\ по
 гравитации и теории относительности} {Изд--во Тбил.\ ун--та}
{Тбилиси} {1968} {} {229--235} {2}

\sbibitem{1968d}
{А.А. Абрикосов, \LD\ и И.М. Халатников}
{Проблемы современной физики}
{\VestnikAN} {12} {} {85--88} {1968}

\sbibitem{1968a}
{D.~Brill, S.~Deser, and \LF}
{Sign of gravitational energy}
{\PL} {A26} {} {538--539} {1968}

\rbibitem{1969a}
{\LD}
{Интеграл Фейнмана для сингулярных лагранжианов}
{\TMF} {1} {no.~1} {3--18} {1969}

\ebibitem{1969a}
{\LF}
{The Feynman integral for singular Lagrangians}
{Theor. Math. Phys.} {1} {1--13} {1969} \\

Reprinted in: \fy{52--64}

\bibitemr{1969b}
{\LD} 
{Квантовая теория калибровочных полей} 
{\em Векторные мезоны и электромагнитные взаимодействия}
{Труды Межд.\ семинара, Дубна, 1969} {JINR} {Дубна} {1969} {} {13-26} {2}

\rbibitem{1970a}
{А.А. Славнов и \LD}
{Безмассовое и массивное поле Янга--Миллса}
{\TMF} {3} {no.~1} {18--23} {1970}

\ebibitem{1970a}
{A.A. Slavnov and L.D. Faddeev}
{Massless and massive Yang--Mills field}
{Theor. Math. Phys.} {3} {312--316} {1970}

\rbibitem{1970b}
{П.П. Кулиш и \LD}
{Асимптотические условия и инфракрасные расходимости в квантовой электродинамике}
{\TMF} {4} {no.~2} {153--170} {1970}

\ebibitem{1970b}
{P.P. Kulish and L.D. Faddeev}
{Asymptotic conditions and infrared divergencies in
 quantum electrodynamics}
{Theor. Math. Phys.} {4} {745--757} {1970} \\

\bibiteme{1970d}
{\LF}
{Recent developments in the integral equations treatment of the
 few-body scattering problem.} 
{\em Three body problem in nuclear and particle physics}
{North Holland} {Amsterdam--London} {1970} {154-167} {3}

\rbibitem{1971f}
{А.А. Славнов и \LD}
{Инвариантная теория возмущений для нелинейных киральных лангражианов}
{\TMF} {8} {no.~3} {297--307} {1971}

\ebibitem{1971f}
{A.A. Slavnov and L.D. Faddeev}
{Invariant perturbation theory for nonlinear chiral Lagrangians}
{Theor. Math. Phys.} {8} {843--850} {1971}

\rbibitem{1971b}
{В.Е. Захаров и \LD}
{Уравнение Кортвега--де Фриса --- вполне интегрируемая гамильтонова система}
{\FA} {5} {no.~4} {18--27} {1971}

\ebibitem{1971b}
{V.E. Zakharov and \LF}
{Korteweg--de Vries equation: a completely integrable
  Hamiltonian system}
{Funct. Anal. Appl.} {5} {280--287} {1971} \\

\rbibitem{1971c}
{\LD}
{К теории устойчивости стационарных плоско--параллельных
течений идеальной жидкости}
{\Zap} {21} {} {164--172} {1971}

\ebibitem{1971c}
{\LF}
{On the theory of the stability of stationary plane parallel flows of
  an ideal fluid}
{\JSM} {1} {518--525} {1973}

\bibitemr{1971d}
{\LD} 
{\em Метод интегральных уравнений в теории рассеяния для трех и более частиц.} 
{Конспекты лекций} {М.: МИФИ} {} {1971} {} {50~с} {4}

\bibitemr{1971e}
{\LD} \\
{Интегральные уравнения теории рассеяния и малонуклонные системы.} \\
{\em Проблемы современной ядерной физики (Сб.\ докл.\ на 2--м
 проблемном симпозиуме по физике ядра, Новосибирск, 1970)} {М.: Наука} {} {1971}{} {5-18} {2}

\bibitemr{1971h}
{В.Н. Попов и \LD}
{Ковариантное квантование гравитационного поля при помощи интеграла Фейнмана.}
{\em Функциональные методы в квантовой теории поля и статистике}
{} {М.} {1971} {ч.~2} {с.~9}

\bibiteme{1971a}
{\LF} 
{Symplectic structure and quantization of the Einstein gravitation theory.}
{\em Actes du Congr\`es International des Math\'ematiciens (Nice, 1970)}
{Gauthier--Villars} {Paris} {1971} {vol.~3} {35--39} {2}

\bibitemr{1971a}
{\LD}
{Симплектическая структура и квантование теории тяготения Эйнштейна}. \\
{\em Международный конгресс математиков (Ницца, 1970)}
{М.: Наука} {} {1972} {328-333} {2}

\bibiteme{1971aa}
{\LF}
{Three-dimensional inverse problem in the quantum theory of scattering.}
{Preprint ITP--71--106E} {ITP} {Kiev} {1971} {} {28~pp} {1}

\bibitemr{1971aa}
{Трехмерная обратная задача квантовой теории рассеяния} 
{\em Обратные задачи для дифференциальных уравнений (Труды Всесоюзн.\ симпоз., Новосибирск, 1971)}
{} {Новосибирск} {1972} {} {14-30} {2}

\rbibitem{1972c}
{Б.С. Павлов и \LD}
{Теория рассеяния и автоморфные функции}
{\Zap} {27} {} {161--193} {1972}

\ebibitem{1972c}
{B.S. Pavlov and \LF}
{Scattering theory and automorphic functions}
{\JSM} {3} {522--548} {1975} \\

\rbibitem{1972d}
{Дифференциальная геометрия и лагранжева механика со связями}
{А.М. Вершик и \LD}
{\DAN} {202} {no.~3} {555--557} {1972}

\ebibitem{1972d}
{A.M. Vershik and \LF}
{Differential geometry and Lagrangian mechanics with constraints}
{\SPD} {17} {34--36} {1972}

\sbibitem{1972e}
{\LF}
{Asymptotic conditions and infrared divergencies in
 quantum electrodynamics} {(New developments in the relativistic
quantum field theory and its applications)
Acta Univ.\ Bratislaviensis} {1} {no.~164} {69--78} {1972}

\rbibitem{1973b}
{В.Н. Попов и \LD}
{Ковариантное квантование гравитационного поля}
{\UFN} {111} {no.~3} {427--450} {1973}

\ebibitem{1973b}
{\LF\ and V.N. Popov}
{Covariant quantization of the gravitational field}
{\SPU} {16} {777--788} {1974} \\

Reprinted in: \fy{65--76}

\rbibitem{1973a}
{А.Б. Венков, В.Л. Калинин и \LD}
{Неарифметический вывод формулы следа Сельберга}
{\Zap} {37} {} {5--42} {1973}

\ebibitem{1973a}
{A.B. Venkov, V.L. Kalinin and \LF}
{A nonarithmetic derivation of the Selberg trace formula}
{\JSM} {8} {171--199} {1977}

\rbibitem{1973c}
{\LD}
{Калибровочно--инвариантная модель электромагнитного и
 слабого взаимодействия лептонов}
{\DAN} {210} {no.~4} {807--810} {1973}

\ebibitem{1973c}
{\LF}
{Gauge--invariant model of electromagnetic and weak lepton interactions}
{\SPD} {18} {382} {1973}

\sbibitem{1973d}
{\LF\ and A.A. Slavnov}
{Higher orders in the invariant perturbation theory for chiral
 Lagrangians}
{Lett.\ Nuovo Cimento} {8} {} {117--120} {1973}

\bibitemr{1974a}
{\LD}
{Обратная задача квантовой теории рассеяния. II}
{\em Современные проблемы математики (Итоги науки и техники)}
{М.: ВИНИТИ} {} {1974} {т.~3} {93--181} {3}

\bibiteme{1974a}
{\LF} 
{The inverse problem in the quantum theory of scattering. II} 
{\JSM} {} {} {1976} {5} {334--396} {1}

Reprinted in: \fy{121--183}

\rbibitem{1974b}
{Л.А. Тахтаджян и \LD}
{Существенно--нелинейная одномерная модель классической теории поля}
{\TMF} {21} {no.~2} {160--174} {1974}

\ebibitem{1974b}
{\LF\ and L.A. Takhtajan}
{Essentially nonlinear one--dimensional model of the classical
 field theory}
{Theor. Math. Phys.} {21} {1046--1057} {1975} \\

\rbibitem{1974e}
{И.Я. Арефьева, А.А. Славнов и \LD}
{Производящий функционал для S--матрицы в калибровочно--инвариантных теориях}
{\TMF} {21} {no.~3} {311--321} {1974}

\ebibitem{1974e}
{I.Ya. Arefeva, \LF, and A.A. Slavnov}
{Generating functional for the S matrix in gauge theories}
{Theor. Math. Phys.} {21} {1165--1172} {1975}

\bibitemr{1974g}
{Л.А. Тахтаджян и \LD}, \\
{Частицы для уравнения Сайн--Гордон}.\\
{\em \UMN} {\bf 28} {no.~3} {(1974)} {249--250}
%{\LF\ and L.A. Takhtajan}
%{}
%{? \RMS} {} {} {}

\bibitemr{1974h}
{\LD}
{Об одном подходе к объединению электромагнитного и
слабого взаимодействия лептонов.}
{\em Труды семинара по $\mu$--e проблеме}
{М.: Наука} {} {1974} {} {158-161} {2}

\rbibitem{1974c}
{В.Е. Захаров, Л.А. Тахтаджян и \LD}
{Полное описание решений ``sine--Gordon'' уравнения}
{\DAN} {219} {no.~6} {1334--1337} {1974}

\ebibitem{1974c}
{V.E. Zakharov, L.A. Takhtadjan and \LF}
{Complete description of solutions of the sine--{G}ordon
  equation}
{\SPD} {19} {824--826} {1975}

\bibiteme{1974d}
{\LF} 
{Vortex--like solutions in a unified model of electromagnetic and
 weak interactions of leptons.} 
{Preprint MPI--PAE/PTh 16} {} {M\"unchen} {1974} {} {9~pp} {1}

\bibitemr{1975c}
{А.М. Вершик и \LD}
{Лагранжева механика в инвариантом изложении.}
{\em Проблемы теоретической физики. (Т.~2. Теория ядра.
 Функциональные методы в квант.\ теории поля и стат.\ физике.
 Мат.\ физика.)} {Л.: Изд--во ЛГУ} {} {1975} {} {129--141} {3}

\bibiteme1{1975c}
{A.M. Vershik and \LF} 
{Lagrangian mechanics in invariant formulation} 
{Selecta Math. Soviet} {\bf 1}, no.~4 (1981), 339--350. \\

%In: {\em Problems of theoretical physics, II, Theory of the atom.
%Functional methods in quantum field theory and statistical physics,
%Mathematical physics.}  (Izdat. Leningrad. Univ.,
%  Leningrad, 1975)
Reprinted in: \fy{}

\rbibitem{1975f}
{Л.А. Тахтаджян и \LD}
{Существенно--нелинейная одномерная модель классической теории поля. (Дополнение)}
{\TMF} {22} {no.~1} {143} {1975}

\ebibitem{1975f}
{\LF\ and L.A. Takhtajan}
{Essentially nonlinear one--dimensional model of the classical
 field theory. (Addendum)}
{Theor. Math. Phys.} {22} {100} {1975}

\rbibitem{1975hh}
{\LD}
{Адроны из лептонов?}
{\LZ} {21} {no.~2} {141--144} {1975}

\ebibitem{1975hh}
{\LF}
{Hadrons from leptons?}
{JETP Letters} {21} {64--65} {1975}

\bibiteme{1975i}
{\LF}
{Quantization of solitons}
{Preprint IAS Print--75--QS70} {Inst.\ Advanced Study} {Princeton, NJ}
{1975} {} {32~pp} {1}

\rbibitem{1975bb}
{В.Е. Корепин, П.П. Кулиш и \LD}
{Квантование солитонов}
{\LZ} {21} {no.~5} {302--305} {1975}

\ebibitem{1975bb}
{\LF, V.E. Korepin and P.P. Kulish}
{Quantization of solitons}
{JETP Letters} {21} {138--139} {1975}

\rbibitem{1975b}
{В.Е. Корепин и \LD}
{Квантование солитонов}
{\TMF} {25} {no.~2} {147--163} {1975}

\ebibitem{1975b}
{V.E. Korepin and \LF}
{Quantization of solitons}
{Theor. Math. Phys.} {25} {1039--1049} {1975}

\rbibitem{1975a}
{\LD}
{Дифференциально--геометрические структуры и квантовая теория поля}
{\Trudy} {135} {} {218--223} {1975}

\ebibitem{1975a}
{\LF}
{Differential--geometric structures, and quantum field theory}
{Proc.\ Steklov Math.\ Inst.} {1} {223--228} {1978}
%{? International Conference on the Mathematical Problems of Quantum
%  Field Theory and Quantum Statistics. Part I. Axiomatic Quantum Field Theory.}

\rbibitem{1975d}
{А.Г. Рейман и \LD}
{Об одном классе бесконечномерных динамических систем}
{\Vestnik} {1} {вып.~1}  {138--142} {1975}

\ebibitem{1975d}
{A.G. Reyman and \LF}
{On a class of infinite--dimensional dynamical systems}
{Vestnik Leningrad Univ., Math.} {8} {145--150} {1980}

\bibitemr{1975ff}
{\LD}
{Эквивалентность канонического и ковариантного подходов к
квантованию асимптотически плоского поля тяготения Эйнштейна.}
{\em Проблемы гравитации (Докл.\ 3--й Сов.\ гравитационной
 конф., Ереван, 1972)} {Ереван: Изд--во Ереванск.\ ун--та} {} {1975} {} {90--103} {2}

\rbibitem{1976d}
{П.П. Кулиш, С.В. Манаков и \LD}
{Сравнение точных квантовых и квазиклассических ответов
 для нелинейного уравнения Шредингера}
{\TMF} {28} {} {38--45} {1976}

\ebibitem{1976d}
{P.P. Kulish, S.V. Manakov, and \LF}
{Comparison of the exact quantum and quasiclassical results
for the nonlinear Schr\"odinger equation}
{Theor. Math. Phys.} {28} {615--620} {1976}

\rbibitem{1976a}
{Л.А. Тахтаджян и \LD}
{Гамильтонова система, связанная с уравнением
 $u\sb{\xi} \sb{\eta }+{\rm sin}\ u=0$}
{\Trudy} {142} {} {254--266} {1976}

\ebibitem{1976a}
{L.A. Takhtajan and \LF}
{The Hamiltonian system connected with the equation
 $u\sb{\xi} \sb{\eta }+{\rm sin}\ u=0$}
{Proc.\ Steklov Inst.\ Math.} {142} {277--289} {1979}

\bibiteme{1976e}
{D. Brill, S. Deser, and \LF} 
{Positive definiteness of gravitational field energy.}
{PGTbilisi} {} {} {1976} {} {32-42} {2}

\bibitemr{1976f}
{В.Н. Попов и \LD,} 
{Континуальный интеграл Фейнмана в теории тяготения.} 
{PGTbilisi} {} {} {1976} {} {500--510} {2}

\bibitemr{1976j}
{\LD} 
{Гамильтонова формулировка теории тяготения Эйнштейна.} 
{PGTbilisi} {} {} {1976} {} {676--688} {2}

\bibitemr{1976b}
{\LD}
{В поисках многомерных солитонов.} 
{\em Нелокальные, нелинейные и неренормируемые теории поля
(Материалы IV Междунар.\ совещ.\ по нелокальным теориям поля,
 Алушта, 1976)} {} {Дубна} 1976} {} {207--223} {2}

%{\LF,} \\
%{In search of multidimensional solitons.} \\
%In: {\em Nonlocal, nonlinear and nonrenormalizable field theories} (Proc.\ of
%  Fourth Internat. Sympos. on Nonlocal Field Theories, Alushta, 1976)
% (Dubna 1976), {207--223} (in Russian). \\
Reprinted in: \fy{369--381}

\bibiteme{1976c}
{\LF}
{Introduction to the functional methods}
{LesHouches75}
{North--Holland} {Amsterdam} {1976} {} {3-40} {2}

Reprinted in: \fy{79--119}

\bibiteme{1976cc}
{\LF} 
{Localized solutions of nonlinear classical field equations
 and their quantum interpretation} 
{LesHouches75}
{North--Holland} {Amsterdam} {1976} {} {253-254} {2}

\sbibitem{1976ee}
{\LF}
{Some comments on the many dimensional solitons}
{Lett. Math. Phys.} {1} {} {289--293} {1976}

\sbibitem{1976ff}
{\LF\ and V.E. Korepin}
{About the zero mode problem in the quantization of solitons}
{\PL} {B63} {} {435--438} {1976}

\bibitemr{1977aa}
{\LD} 
{Метод обратной задачи рассеяния для решения эволюционных
 уравнений математической физики}
{Материалы VIII Дальневост. мат.\ школы, Владивосток, 1976}
{АН СССР. ДВНЦ, Хабаровск.\ комплексный НИИ} {} {1977}
{Препринт no.~1-77}
{39~с} {2}

% {\LF,} \\
%{The method of the inverse scattering problem for the solution of
% evolution equations in mathematical physics.}\\
% {\em Materials of the VIIIth Far--Eastern Mathematical School,
% Vladivostok, 1976},
% Preprint no.~1 (Khabarovsk, 1977), 37~pp.

\rbibitem{1977b}
{М.А. Семенов--Тян--Шанский и \LD}
{К теории нелинейных киральных полей}
{\Vestnik} {13} {вып.~3} {81--88} {1977}

\ebibitem{1977b}
{M.A. Semenov--Tian--Shansky and \LF}
{On the theory of nonlinear chiral fields}
{Vestnik Leningrad Univ., Math.}  {10} {319--327} {1982}

\bibiteme{1977i}
{\LF} 
{Quantization of solitons} 
{\em Труды Международной конференции по физике высоких энергий. (Тбилиси, 1976).} {JINR} {Дубна} {1977} {т.~2} {751--752} {2}

\bibitemr{1977f}
{В.Е. Корепин и \LD}
{Квантование солитонов.}
{\em Физика элементарных частиц (Материалы XII Зимней школы ЛИЯФ).}
{} {Л.} {1977} {} {130-146} {2}

\sbibitem{1978c}
{\LF\ and V.E. Korepin}
{Quantum theory of solitons}
{Physics Reports} {C42} {no.~1} {1--87} {1978}

\sbibitem{1978b}
{\LF\ and P.P. Kulish}
{Quantization of particle--like solutions in field theory}
{Lecture Notes in Phys.} {80} {} {270--278} {1978}

\rbibitem{1978d}
{Е.К. Склянин и \LD}
{Квантовомеханический подход к вполне интегрируемым моделям теории поля}
{\DAN} {243} {no.~6} {1430--1433} {1978}

\ebibitem{1978d}
{E.K. Sklyanin and \LF}
{Quantum mechanical approach to completely integrable field theory models}
{\SPD} {23} {902--904} {1978}

\rbibitem{1978h}
{Б.С. Павлов и \LD}
{Нуль--множества операторных функций с положительной мнимой частью}
{\Zap} {81} {}  {85--88} {1978}

\ebibitem{1978h}
{B.S. Pavlov and \LF}
{Zero sets of operator functions with a positive imaginary part}
{Lecture Notes in Math.} {1043} {124--128} {1984}

\sbibitem{1978ee}
{\LD}
{Работы В.А. Фока по математической физике}
{Труды оптического ин--та (ГОИ)} {43} {вып.~177} {37--39} {1978}

\bibiteme{1978j}
{\LF} 
{Introduction to the functional methods in quantum field theory.} 
{\em Proc. of the III School on elementary particles and
 high energy physics. (Primorsko, Bulgaria, 1977).}
{} {Sofia} {1978} {} {193-238} {2}

\bibitemr{1979d}
{\LD}
{Квантовые вполне интегрируемые модели теории поля.} 
{Препринт ЛОМИ Р--79--02} {} {} {1979} {} {57 c.} {1}

\bibitemr{1979d}
{\LD}
{Квантовые вполне интегрируемые модели теории поля.} 
{\em Проблемы квантовой теории поля
(Материалы V междунар.\ совещ.\ по нелокальным теориям поля,
  Алушта, 1979} {} {Дубна} {1979} {} {249-299} {2}

%{\LF,} \\
%{Quantum completely integrable models of field theory.} \\
%Preprint LOMI P--79--02 (1979), 57~pp. \\
%And in: {\em Problems of quantum field theory} (Proc.\ of Alushta, 1979), 249--299.

\rbibitem{1979b}
{Е.К. Склянин,  Л.А. Тахтаджян и \LD}
{Квантовый метод обратной задачи.~I}
{\TMF} {40} {no.~2} {194--220} {1979}

\ebibitem{1979b}
{E.K. Sklyanin, L.A. Takhtajan, and \LF}
{Quantum inverse problem method.~I}
{Theor. Math. Phys.} {40} {688--706} {1979} \\

\rbibitem{1979a}
{Л.А. Тахтаджян и \LD}
{Квантовый метод обратной задачи и XYZ--модель Гейзенберга}
{\UMN} {34} {no.~5} {13--63} {1979}

\ebibitem{1979a}
{L.A. Takhtajan and \LF}
{The quantum method for the inverse problem and the XYZ
  Heisenberg model}
{\RMS} {34} {11--68} {1979} \\

Reprinted in: \fy{}

\rbibitem{1979c}
{А.Г. Изергин, В.Е. Корепин, М.А. Семенов--Тян--Шанский и \LD}
{О калибровочных условиях для поля Янга--Миллса}
{\TMF} {38} {no.~1} {3--14} {1979}

\ebibitem{1979c}
{A.G. Izergin, V.E. Korepin, M.A. Semenov--Tian--Shansky, and \LF}
{Gauge conditions for a Yang--Mills field}
{Theor. Math. Phys.} {38} {1--9} {1979}

\bibiteme{1979e}
{\LF} 
{Einstein and several contemporary tendencies in the theory
of elementary particles.} 
{\em Relativity, quanta, and cosmology in the development
of the scientific thought of A.~Einstein}
{Johnson Repr. Corp.} {NY} {1979} {vol. 1}  {247-266} {3}

\bibitemrp{1979e}
{en} 
{Astrofisica e cosmologia, gravitazione, quanti e
relativita. Negli sviluppi del pensiero scientifico di A.~Einstein}
{Giunti Barbera} {Firenze} {1979} {} {765--793} {3}

Reprinted in: \fy{441--461}

\bibiteme{1980a}
{\LF}
{Quantum completely integrable models in field theory.}
{\em Mathematical physics reviews. Sect. C.: Mathematical physics}
{Harwood Acad. Publ.} {Chur} {1980} {vol. 1} {107--155} {3}

%{Soviet Sci. Rev.} {C1} {} {107--155} {1980} \\
Reprinted in: \fy{187--235}

\bibiteme{1980c}
{\LF}
{A Hamiltonian interpretation of the inverse scattering method.} 
{Solitons} {Springer} {Berlin-NY} {1980} {339-354} {3}
%Bullough, R.K., Caudrey, P.J.:

\bibitemr{1980c}
{\LD} {A Hamiltonian interpretation of the inverse scattering method}
{Солитоны} {М.: Мир} {} {1983} {} {363--379} {3} 

\bibiteme{1980g}
{\LF and P.P. Kulish}
{Development of the quantum inverse problem method}
{\em Workshop on nonlinear evolution equations and
 dynamical systems. (Crete, 1980)} {Orthodox Acad. of Crete}
{Chania} {1980} {} {121} {2}

\bibitemr{1977a}
{\LD}
{Квантование солитонов.} 
{Препринт} {Владивосток: АН СССР. ДВНЦ, Хабаровск. комплексный НИИ}
{} {1981} {} {42~с} {1}

\sbibitem{1981d}
{\LF}
{Two--dimensional integrable models in quantum field theory}
{Physica Scripta} {24} {no.~5} {832--835} {1981}

\sbibitem{1981e}
{\LF\ and L.A. Takhtajan}
{What is the spin of a spin wave?}
{\PL} {A85} {} {375--377} {1981}

\rbibitem{1981f}
{А.М. Веселова и \LD}
{Особенность в кулоновском трехчастичном рассеянии на пороге ионизации}
{\VestnikF} {22} {вып.~4} {42--46} {1981}

\sbibitem{1981f}
{A.M. Veselova and \LF}
{A singularity in Coulomb three--particle
scattering on the threshold of ionization}
{Studia Logica} {40} {no.~3} {42--46} {1981}

\rbibitem{1981c}
{\LD\ и О.А. Якубовский}
{О мнимых парадоксах в теории рассеяния нескольких частиц}
{\YF} {33} {вып.~3} {634--636} {1981}

\ebibitem{1981c}
{\LF\ and O.A. Yakubovski}
{On imaginary paradoxes in few--particle scattering theory}
{Soviet J.\ Nucl.\ Phys.} {33} {331--332} {1981}

\rbibitem{1981g}
{Л.А. Тахтаджян и \LD}
{Спектр и рассеяние возбуждений в одномерном изотропном магнетике Гейзенберга}
{\Zap} {109} {} {134--178} {1981}

\ebibitem{1981g}
{L.A. Takhtajan and \LF}
{The spectrum and scattering of excitations in the one--dimensional
  isotropic Heisenberg model}
{\JSM} {24} {241--267} {1984}

\rbibitem{1982b}
{\LD}
{Проблема энергии в теории тяготения Эйнштейна}
{\UFN} {136} {} {435--457} {1982}

\ebibitem{1982b}
{\LF}
{The energy problem in Einstein's theory of gravitation}
{\SPU} {25} {130--142} {1982}

\rbibitem{1982a}
{Л.А. Тахтаджян и \LD}
{Простая связь геометрического и гамильтонова представлений
 интегрируемых нелинейных уравнений}
{\Zap} {115} {} {264--273} {1982}

\ebibitem{1982a}
{L.A. Takhtajan and \LF}
{A simple connection between geometric and Hamiltonian
  representations of integrable nonlinear equations}
{\JSM} {28} {800--806} {1985}

\bibiteme{1982g}
{\LF}
{Recent development of quantum spectral transform (QST).} 
{\em Recent development in gauge theory and integrable systems}
{Kyoto Univ., RIMS} {} {1982} {} {53-71} {3}

\bibitemrp{1982g}
{en}
{VII Brazilian symposium on theoretical physics, Rio de Janeiro, 1982}
{CNPq} {Brasilia} {1984} {} {257--274} {2}

\bibiteme{1981h}
{\LF}
{Quantum scattering transformation.}
{\em Structural elements in particle physics and
 statistical mechanics (NATO ASI series. Ser. B: Physics)}
{Plenum Press} {NY-London} {1983} {82} {93--114} {1}

\rbibitem{1983f}
{\LD}
{Замечание о статье В.И. Денисова, В.О. Соловьева}
{\TMF} {56} {no.~2} {315--316} {1983}

\ebibitem{1983f}
{\LF}
{Note on the paper by V.I. Denisov and V.O. Solov'ev}
{Theor. Math. Phys.} {56} {842} {1983}

\rbibitem{1983c}
{Н.Ю. Решетихин и \LD}
{Гамильтоновы структуры для интегрируемых моделей теории поля}
{\TMF} {56} {no.~3} {323--343} {1983}

\ebibitem{1983c}
{N.Yu. Reshetikhin and \LF}
{Hamiltonian structures for integrable field theory models}
{Theor. Math. Phys.} {56} {847--862} {1983}

\rbibitem{1983a}
{В.О. Тарасов, Л.А. Тахтаджян и \LD}
{Локальные гамильтонианы для интегрируемых квантовых моделей на решетке}
{\TMF} {57} {no.~2} {163--181} {1983}

\ebibitem{1983a}
{V.O. Tarasov, L.A. Takhtajan, and \LF}
{Local Hamiltonians for integrable quantum models on a lattice}
{Theor. Math. Phys.} {57} {1059--1073} {1983}

\bibiteme{1983g}
{\LF and L.A. Takhtajan}
{Integrability of quantum O(3) nonlinear $\sigma$-model.}
{Preprint LOMI} {} {} {1983} {E-4-83} {19pp.} {1}

\bibiteme{1983e}
{\LF}
{Integrable models in (1+1)--dimensional quantum field theory}
{\em Recent advances in field theory and statistical mechanics}
(Proc.\ of Les Houches, Session XXXIX, 1982)}
{North-Holland} {Amsterdam} {1984} {} {561--608} {2}

Reprinted in: \fy{294--341}

\bibiteme{1984k}
{\LF}
{General properties of field theory: Introductory remarks}
{\em High energy physics (Proc.\ of 22th Int.\ conf.\ on
high energy physics, Leipzig, 1984}
{} {Berlin-Zeuthen} {1984} {vol.1} {121} {2}

\sbibitem{1984m}
{\LF}
{Operator anomaly for the Gauss law}
{\PL} {B145} {} {81--84} {1984}

\bibitemr{1984f}
{Н.Ю. Решетихин и \LD} 
{Интегрируемость квантовой модели главного кирального поля}
{\em Труды VII Междунар.\ совещ.\ по проблемам квантовой теориям поля,
 Алушта, 1984} {JINR} {Дубна} {1976} {} {37--55} {2}

%{\LF\ and N.Yu. Reshetikhin,} \\
%{Integrability of a quantum model of a principal chiral field.} \\
%In: {Proc.\ of the VII international conference on the problems of
%  quantum field theory, Alushta, 1984} (Dubna, 1984), {37--55}.

\rbibitem{1984kk}
{\LD\ и С.Л. Шаташвили}
{Алгебраические и гамильтоновы методы в теории неабелевых аномалий}
{\TMF} {60} {no.~2} {206--217} {1984}

\ebibitem{1984kk}
{\LF\ and S.L. Shatashvili}
{Algebraic and Hamiltonian methods in the theory of nonabelian
  anomalies}
{Theor.\ Math.\ Phys.} {60} {770--778} {1985}

\rbibitem{1984h}
{А.Г. Рейман, М.А. Семенов--Тян--Шанский и \LD}
{Квантовые аномалии и коциклы на калибровочных группах}
{\FA} {18} {no.~4} {64--72} {1984}

\ebibitem{1984h}
{A.G. Reyman, M.A. Semenov--Tian--Shansky, and \LF}
{Quantum anomalies and cocycles on gauge groups}
{Funct.\ Anal.\ Appl.} {18} {319--326} {1984}

\sbibitem{1984j}
{\LF\ and N.Yu. Reshetikhin}
{Evaluation of an infinite product of special matrices}
{Lecture Notes in Math.} {1043} {} {177--179} {1984}

\sbibitem{1985c}
{\LF}
{Classical and quantum {$L$}--matrices}
{Lecture Notes in Phys.} {242} {} {158--174} {1985}

\rbibitem{1985e}
{\LD}
{Коциклы группы токов и квантовая теория полей Янга--Миллса}
{\UMN} {40} {no.~4} {117--120} {1985}

\ebibitem{1985e}
{\LF}
{Cocycles of the current group and the quantum theory of
  Yang--Mills fields}
{\RMS} {40} {no.~4, 129--133} {1985}

\sbibitem{1985d}
{\LF\ and L.A. Takhtajan}
{Poisson structure for the {K}d{V} equation}
{Lett. Math. Phys.} {10} {} {183--188} {1985}

\sbibitem{1986d}
{V.S. Buslaev, \LF, and L.A. Takhtajan}
{Scattering theory for the {K}orteweg--de Vries (KdV) equation
  and its {H}amiltonian interpretation}
{Physica} {D18} {} {255--266} {1986}

\bibiteme{1986b}
{\LF}
{Can theories with anomalies be quantized?}
{Supersymmetry and its applications: superstrings, anomalies, and
 supergravity} {Cambridge Univ. Press} {Cambridge} {1986} {} {41-53} {3}

\sbibitem{1986e}
{\LF\ and N.Yu. Reshetikhin}
{Integrability of the principal chiral field model in {$1+1$} dimension}
{Ann. Physics} {167} {} {227--256} {1986}

\sbibitem{1986c}
{\LF\ and L.A. Takhtajan}
{Liouville model on the lattice}
{Lecture Notes in Phys.} {246} {} {166--179} {1986}

\sbibitem{1986g}
{\LF\ and  S.L. Shatashvili}
{Realization of the Schwinger term in the Gauss law
and the possibility of correct quantization of a
theory with anomalies}
{\PL} {B167} {} {225--228} {1986}

\rbibitem{1987a}
{А.Ю. Алексеев, Я. Мадайчик, \LD\ и С.Л. Шаташвили}
{Вывод аномального коммутатора в формализме функционального интеграла}
{\TMF} {73} {no.~2} {187--190} {1987}

\ebibitem{1987a}
{A.Yu. Alekseev, Ya. Mada{i}chik, \LF, and S.L. Shatashvili}
{Derivation of an anomalous commutator in the formalism of a
  functional integral}
{Theor.\ Math.\ Phys.} {73} {1149--1151} {1988}

\bibitemr{1987aa}
{А.М. Веселова, С.П. Меркурьев и \LD} 
{Кулоновская S--матрица и многократное рассеяние} \\
{Дифракционное взаимодействие адронов с ядрами}
{Киев: АН УССР. Инст.\ теорет.\ физики} {} {1987} {} {107--114} {3}

\bibiteme{1987g}
{\LF}
{Hamiltonian approach to the theory of anomalies}
{\em Recent Development in Mathematical Physics
(Proc.\ of XXVI Int.\ Universit\"atswochen f\"ur
Kernphysik, Schladming, Austria, 1987)}
{Springer-Verlag} {Berlin} {1987} {} {137-159} {2}

Reprinted in \fy{385--407}

\rbibitem{1987h}
{\LD}
{Тридцать лет в математической физике}
{\Trudy} {176} {} {4--29} {1987}

\ebibitem{1987h}
{\LF}
{Thirty years in mathematical physics}
{Proc.\ Steklov Inst.\ Math.} {176} {3--28} {1988} \\
Reprinted in \fy{3--28}

\sbibitem{1988d}
{L.Faddeev and R.~Jackiw}
{Hamiltonian reduction of unconstrained and constrained systems}
{Phys. Rev. Lett.} {60} {} {1692--1694} {1988}

\sbibitem{1988a}
{A.~Alekseev, L.~Faddeev, and S.~Shatashvili}
{Quantization of symplectic orbits of compact {L}ie groups by means of
  the functional integral}
{J. Geom. Phys.} {5} {} {391--406} {1988}

\bibiteme{1988b}
{\LF, N.Yu. Reshetikhin, and L.A. Takhtajan,}\\
{Quantization of Lie groups and Lie algebras} \\
{Algebraic analysis} {Academic Press} {Boston, MA} {1988} {vol.1}
 {129--139} {1}

\bibitemrp{1988b}
{en}
{Yang-Baxter equation in integrable systems}
{World Sci. Publ.} {Singapore} {1990} {} {299--309} {3}

\bibiteme{1989b}
{L.~Faddeev, N.~Reshetikhin, and L.~Takhtajan} 
{Quantum groups}
{Braid group, knot theory and statistical mechanics}
{World Sci. Publ.} {Singapore} {1989} {} {97--110} {3}

\sbibitem{1989a}
{\LF}
{Quantum groups}
{Bull.\ Brasil Math.\ Soc.\ (N.S.)} {20} {no.~1} {47--54} {1989} 

\rbibitem{1989c}
{Н.Ю. Решетихин, Л.А. Тахтаджян и \LD}
{Квантование групп Ли и алгебр Ли}
{Алгебра и анализ} {1} {no.~1} {178--206} {1989}

\ebibitem{1989c}
{N.Yu. Reshetikhin, L.A. Takhtajan, and \LF}
{Quantization of {L}ie groups and {L}ie algebras}
{Leningrad Math. J.} {1} {193--225} {1990}

\bibitemr{1989d}
{\LD}
{Математический взгляд на эволюцию физики}
{Природа} {} {} {1989} {no.~5} {11-16} {1}

\bibiteme{1989d}
{\LF}
{A mathematician's view of the development of physics.}  
{\em Frontiers in physics, high technology, and mathematics (Proc.\ of Trieste, 1989)}
{World Sci. Publ.} {Singapore} {1990} {} {238-246} {3}

\bibitemrp{1989d}
{en}
{Miscellanea Mathematica} {Springer--Verlag} {Berlin--Heidelberg}
{1991} {} {119--127} {3}

\bibitemrp{1989d}
{en}
{Les relations entre les math\'ematiques
 et la physique th\'eorique}
{Inst.\ Hautes \'Etudes Sci.} {Bures--Sur--Yvette} {1998} {} {73--79} {3}

%\bibitem{1993e}
%{\LF,} \\
%{The end of physics as seen by a mathematician.} (Finnish transl.) \\
%In: {\em In the forest of symbols}
%(Art House, Helsinki, 1992), {231--243}.

\bibiteme{1990d}
{\LF}
{Lectures on quantum inverse scattering method}
{\em Integrable systems (Nankai Lectures Math. Phys., Tianjin, 1987)}
{World Sci. Publ.} {Singapore} {1990} {} {23-70} {3}

%\bibitem{1990b} ??
%{L. Faddeev,} \\
%{Lectures on quantum inverse scattering method.} \\

\bibitemrp{1990d}
{en}
{New problems, methods and techniques in quantum field theory and statistical mechanics}
{World Sci. Publ.} {Singapore} {1990} {} {7--54} {3}

\sbibitem{1990g}
{\LF}
{On the exchange matrix for {WZNW} model}
{Commun. Math. Phys.} {132} {} {131--138} {1990}

\bibiteme{1991g}
{\LF}
{From integrable models to quantum groups}
{\em Fields and particles (Proc.\ of XXIX Int.\ Universit\"atswochen f\"ur
 Kernphysik, Schladming, Austria, 1990)
{Springer} {Berlin} {1990} {} {89-116} {2}

\sbibitem{1991c}
{A.Yu. Alekseev and \LF}
{$(T\sp *G)\sb t$: a toy model for conformal field theory}
{Commun. Math. Phys.} {141} {} {413--422} {1991}

\bibiteme{1991h}
{A. Yu. Alekseev, \LF, M.A. Semenov--Tian--Shansky, A.Yu. Volkov} 
{The unravelling of the quantum group structure in the WZNW theory} 
{Preprint CERN--TH--5981--91} {} {} {1991} {} {16 pp.} {1}

\sbibitem{1992e}
{A.~Alekseev, L.~Faddeev, and M.~Semenov--Tian--Shansky}
{Hidden quantum groups inside Kac--Moody algebra}
{Commun. Math. Phys.} {149} {} {335--345} {1992}

\rbibitem{1992d}
{А.Ю. Алексеев и \LD}
{Инволюция и динамика {$q$}--деформированного квантового волчка}
{\Zap} {200} {} {3--16} {1992}

\ebibitem{1992d}
{A.Yu. Alekseev and \LF}
{Involution and dynamics for the {$q$}--deformed quantum top}
{J. Math.\ Sci.} {77} {3137--3145} {1995}
\arx{hep-th/9406196}

\bibiteme1{1992a}
{\LF}
{Hamiltonian methods in conformal field theory.}
{\em Mathematical physics, X (Proc.\ of Xth Congr.\ on math.\ phys.,
 Leipzig, 1991}  {Springer} {Berlin} {1992} {} {123-135} {2} {}

\bibiteme{1992c}
{\LF}
{Quantum symmetry in conformal field theory by Hamiltonian methods}
{\em New symmetry principles in quantum field theory %(Carg\`ese, 1991),
(NATO ASI series. Ser. B: Physics)}
{Plenum} {NY-London} {1992} {295} {159-175} {1}

\bibitemrp{1992c}
{en}
{Braid group, knot theory and statistical mechanics, II}
{World Sci. Publ.} {River Edge, NJ} {1994} {} {108--129} {3}

\bibiteme{1992f}
{\LF}
{Integrable models, quantum groups and conformal field theory}
{Preprint SFB288--01} {Freie Univ.} {Berlin} {1992} {} {36~pp} {1}

\rbibitem{1992b}
{А.Ю. Волков и \LD}
{Квантовый метод обратной задачи на дискретном пространстве--времени}
{\TMF} {92} {no.~2} {207--214} {1992}

\ebibitem{1992b}
{A.Yu. Volkov and \LF}
{Quantum inverse scattering method on space--time lattice}
{Theor. Math. Phys.} {92} {837--842} {1992}

\bibiteme{1993b}
{\LF}
{From integrable models to conformal field theory via quantum groups}
{\em Integrable systems, quantum groups, and quantum field theories
(NATO ASI series. Ser.~C: Math.\ phys.\ sci.)}
{Kluwer} {Dordrecht} {1993} {409} {} {1--24} {1}

Reprinted in:  \fy{342--365}

\bibiteme{1993f}
{\LF}
{The Bethe ansatz (Andrejewski lectures)}
{Preprint SFB288--70} {Freie Univ.} {Berlin} {1993} {} {39~pp.} {1}

\sbibitem{1993d}
{L.~Faddeev and A.Yu. Volkov}
{Abelian current algebra and the Virasoro algebra on the lattice}
{\PL} {B315} {} {311--318} {1993}
\arx{hep-th/9307048}

\sbibitem{1994d}
{L.~Faddeev and A.Yu. Volkov}
{Hirota equation as an example of an integrable symplectic map}
{Lett. Math. Phys.} {32} {} {125--135} {1994}
\arx{hep-th/9405087}

\sbibitem{1994dd}
{L.~Faddeev and A.Yu. Volkov}
{The new results on lattice deformation of current algebra}
{Lect. Notes Phys.} {436} {} {1--10} {1994}

\sbibitem{1994e}
{\LF\ and R.M. Kashaev}
{Quantum dilogarithm}
{Modern Phys. Lett.} {A9} {} {427--434} {1994}
\arx{hep-th/9310070}

\bibiteme{1994h}
{\LF, G.P. Korchemsky, and L.N. Lipatov}
{Multi--color QCD at high energies and one--dimensional
Heisenberg magnet}
{Continuous advances in QCD} {} {Minneapolis} {1994} {} {32-41} {3}

\sbibitem{1994j}
{\LF\ and G.P. Korchemsky}
{High energy QCD as a completely integrable model}
{Phys.\ Lett.} {B342} {} {311--322} {1995}
\arx{hep-th/9404173}

\sbibitem{1995j}
{\LF}
{Instructive history of the quantum inverse scattering method}
{Acta Appl. Math.} {39} {} {69--84} {1995}

\sbibitem{1995l}
{\LF}
{Algebraic aspects of the {B}ethe ansatz}
{Int. J. Modern Phys.} {A10} {} {1845--1878} {1995}
\arx{hep-th/9404013}

\sbibitem{1995i}
{\LF}
{Discrete Heisenberg--Weyl group and modular group}
{Lett. Math. Phys.} {34} {} {249--254} {1995}
\arx{hep-th/9504111}

\rbibitem{1995e}
{А.Ю. Волков и \LD}
{Янг--Бакстеризация квантового дилогарифма}
{\Zap} {224} {} {146--154} {1995}

\ebibitem{1995e}
{A.Yu. Volkov and \LF}
{Yang--Baxterization of quantum dilogarithm}
{J.\ Math.\ Sci.} {88} {202--207} {1998}

\sbibitem{1995k}
{\LF\ and R.M. Kashaev}
{Generalized Bethe ansatz equations for Hofstadter problem}
{Commun. Math. Phys.} {169} {} {181--191} {1995}
\arx{hep-th/9312133}

\sbibitem{1995g}
{\LF\ and O.~Tirkkonen}
{Connections of the {L}iouville model and {XXZ} spin chain}
{Nuclear Phys.} {B453} {} {647--669} {1995}
\arx{hep-th/9506023}

\bibiteme{1994g}
{\LF}
{Integrable models on space--time lattice}
{PXICMP} {Internat. Press} {Cambridge, MA} {1995} {} {513--520} {2}

\bibiteme{1995c}
{\LF}
{Traces of integrability in high energy QCD}
{PXICMP} {Internat. Press} {Cambridge, MA} {1995} {} {722} {2}

\bibiteme1{1996c}
{L.~Faddeev}
{Current--like variables in massive and massless integrable models} 
{\em Quantum groups and their applications in physics
(Proc.\ Varenna, 1994)}
{IOS Press} {Amsterdam} {1996} {} {117--135} {2}
{hep-th/9408041}

\sbibitem{1996b}
{A.G. Bytsko and \LF}
{$(T\sp *{B})\sb q,\ q$--analog of model space and the
  Clebsch--Gordan coefficients generating matrices}
{J. Math. Phys.} {37} {} {6324--6348} {1996}
\arx{q-alg/9508022}

\sbibitem{1997e}
{\LF}
{Large new applications of {B}ethe ansatz}
{Lecture Notes in Phys.} {469} {} {51--70} {1996}
%{In Low--dimensional models in statistical physics and quantum field
% theory (Schladming, 1995), Springer, Berlin. .

\bibiteme1{1996d}
{\LF\ and P.N. Pyatov}
{The differential calculus on quantum linear groups}
{\em Contemporary mathematical physics (AMS Transl.)} 
{AMS} {Providence, RI} {1996} {Ser. 2, vol.~175} {35-47} {1}
{hep-th/9402070}

\bibiteme1{1996f}
{\LF} 
{How algebraic Bethe ansatz works for integrable models} 
{\em Sym\'etries quantiques (Proc.\ of Les Houches, Session LXIV 1995)}
{North--Holland} {} {1998} {} {149--219} {2}
{hep-th/9605187}

\bibiteme1{1997b}
{L.~Faddeev and A.~Volkov} 
{Quantum integrable models on $1+1$ discrete space time} 
{\em Quantum fields and quantum space time % (Carg\`ese, 1996),
(NATO ASI series. Ser.~B: Physics)} {Plenum} {NY} {1997} {1}
{364} {73--91} {}

\bibiteme1{1997c}
{L.~Faddeev and A.~Volkov}
{Shift operator for the discrete ${\rm SL}(2)$ current algebra}
{\em Deformation theory and symplectic geometry % (Ascona, 1996),
 (Mathematical Physics Studies)}
{Kluwer Acad. Publ.} {Dordrecht} {1997} {20} {35--42} {1} {}

\bibiteme1{1997dd}
{L.~Faddeev and A.~Volkov} 
{Shift operator for nonabelian lattice current algebra}
{arxiv} {} {} {1996} {} {} {1} {hep-th/9606088}

\bibitem{1997dd}
Modified version published in:
{\em Publ. Res. Inst. Math. Sci.} {\bf 40}, % no.~4,
{1113--1125} (Kyoto U., RIMS, 2004).

\sbibitem{1997g}
{\LF\ and A.J. Niemi}
{Knots and particles} 
{Nature} {387} {} {58--61} {1997}
\arx{hep-th/9610193}

\bibiteme1{1997ee}
{\LF\ and A.J. Niemi}
{Toroidal configurations as stable solitons} 
{arxiv} {} {} {1997} {} {21 pp.} {1} {hep-th/9705176}

\sbibitem{1998g}
{\LF\ and A.J. Niemi}
{Knots as solitons}
{APCTP Bull.} {1} {} {18--22} {1998}

\sbibitem{1998d}
{A.Yu. Alekseev, \LF, J.~Fr{\"o}hlich, and V.~Schomerus}
{Representation theory of lattice current algebras}
{Commun. Math. Phys.} {191} {} {31--60} {1998}
\arx{q-alg/9604017}

\bibiteme1{1999c}
{\LF}
{Instructive history of the quantum inverse scattering method}
{\em Quantum field theory: perspective and prospective
 (Proc.\ of Les Houches, 1998),
(NATO ASI series. Ser.~C: Math.\ phys.\ sci.)}
{Kluwer Acad. Publ.} {Dordrecht} {1999} {530} {161--176} {2} {}

\sbibitem{1999d}
{\LF\ and A.J. Niemi}
{Decomposing the {Y}ang--{M}ills field}
{\PL} {B464} {} {90--93} {1999}
\arx{hep-th/9907180}

\sbibitem{1999h}
{\LF\ and A.J. Niemi}
{Partial duality in SU(N) Yang--Mills theory}
{\PL} {B449} {} {214--218} {1999}
\arx{hep-th/9812090}

\sbibitem{1999i}
{\LF and A.J. Niemi}
{Partially dual variables in SU(2) Yang--Mills theory}
{Phys.\ Rev.\ Lett.} {82} {} {1624--1627} {1999}
\arx{hep-th/9807069}

\bibiteme1{1999f}
{\LF}
{Elementary introduction to quantum field theory}
{Quantum fields and strings: a course for mathematicians}
{AMS} {Providence, RI}
{1999} {vol.~1} {513--550} {3} {}

\bibiteme1{1999g}
{\LF\ and A.Yu. Volkov} 
{Algebraic quantization of integrable models in discrete space--time}
{Discrete integrable geometry and physics}
{Clarendon Press} {Oxford} {1999} {} {301--319} {3}
{hep-th/9710039}

% (Vienna, 1996), {16 of Oxford Lecture Ser. Math. Appl.}

\rbibitem{1999a}
{\LD}
{Что такое современная матматическая физика?}
{\Trudy} {226} {} {7--10} {1999}

\ebibitem{1999a}
{\LF}
{Modern mathematical physics: What is it?}
{Proc.\ Steklov Inst.\ Math.} {226} {1--4} {1999}

\bibiteme1{2000h}
{\LF} 
{Modern mathematical physics: what it should be} 
{Mathematical physics 2000. International congress, London, GB, 2000.}
{Imperial College Press} {London} {2000} {} {1--8} {3}
{math-ph/0002018}

\bibiteme1{2000h}
Reprinted in: {\em Mathematical events of the twentieth century.}
(Springer, Berlin, 2006), 75--84.

\bibiteme1{1990e}
{\LF}
{On the relation between mathematics and physics} 
{Integrable systems (Nankai Lectures Math. Phys., Tianjin, 1987)}
{World Sci. Publ.} {Singapore} {1990} {} {3--9} {3} {}

\bibiteme1{2000a}
{\LF} 
{Modular double of a quantum group} 
{Quantization, deformation, and symmetries. Mathematical Physics Studies} 
{Kluwer Acad.\ Publ.} {Dordrecht}
{2000} {21} {149--156} {3}
{math/9912078}

\sbibitem{2000j}
{\LF\ and A.J. Niemi}
{Magnetic geometry and the confinement of electrically conducting plasmas}
{Phys.\ Rev.\ Lett.} {85} {} {3416--3419} {2000}
\arx{physics/0003083}

\bibiteme1{2000k}
{\LF}
{From Yang--Mills field to solitons and back again}
{\em From the Planck length to the Hubble radius
 (Proc.\ of XXXVI Summer school on subnuclear physics,
 Erice, Italy, 1998)} {} {} {1998} {} {673--685} {2}
{hep-th/9901037}
% XXXVI Summer School on Subnuclear Physics at Erice in September 1998.

\sbibitem{2001b}
{\LF, R.M. Kashaev and A.Yu. Volkov}
{Strongly coupled quantum discrete {L}iouville theory. I.
  Algebraic approach and duality}
{Commun.\ Math.\ Phys.} {219} {} {199--219} {2001}
\arx{hep-th/0006156}

\sbibitem{2001a}
{\LF}
{Knotted solitons and their physical applications}
{Roy.\ Soc.\ Lond.\ Philos.\ Trans., Ser.\ A Math.\ Phys.\ Eng.\ Sci.} {359} {}
 %no. 1784,  {Topological methods in the physical sciences (London, 2000).
{1399--1403} {2001}

\bibiteme1{2001e}
{\LF}
{Quantizing the Yang--Mills fields} 
{\em At the frontier of particle physics. Handbook of QCD.
(Boris Ioffe Festschrift)}
{World Sci. Publ.} {Singapore} {2001} {Vol.~1--3} {80--88} {3} {}

\bibiteme1{2002a}
{\LF}
{Knotted solitons}
{\em Proc.\ of the Int.\ Congr.\ of Mathematicians (Beijing, 2002)}
{Higher Education Press} {Beijing} {2002} {vol.~1} {235--244} {2}
{math-ph/0212079}

\sbibitem{2002b}
{\LF}
{Mass in quantum Yang--Mills theory (comment on a Clay millenium
  problem)}
{Bull.\ Brazil Math.\ Soc.\ (N.S.)} {33} {no.~2} {201--212} {2002}
 
\sbibitem{2002d}
{\LF\ and R.M. Kashaev}
{Strongly coupled quantum discrete Liouville theory. II
  Geometric interpretation of the evolution operator}
{J.\ Phys.} {A35} {} {4043--4048} {2002}
\arx{hep--th/0201049}

\sbibitem{2002e}
{\LF, L.~Freyhult, A.J. Niemi, and P.~Rajan}
{Shafranov's virial theorem and magnetic plasma confinement}
{J.\ Phys.} {A35} {} {L133--L139} {2002}
\arx{physics/0009061}

\sbibitem{2002f}
{\LF\ and A.J. Niemi}
{Aspects of electric and magnetic variables in SU(2)
  Yang--Mills theory}
{\PL} {B525} {} {195--200} {2002}
\arx{hep--th/0101078}

\bibiteme1{2000m}
{\LF}
{Advent of the Yang--Mills field}
{\em Highlights of mathematical physics
(Proc.\ of Congr.\ Math.\ Phys., London, 2000)}
{AMS} {Providence, RI}
{2002} {} {133--141} {2} {}

\sbibitem{2002h}
{E. Babaev, \LF, and A.J. Niemi}
{Hidden symmetry and knot solitons in a charged
two--condensate Bose system}
{Phys.\ Rev.} {B65} {} {100512} {2002}
\arx{cond--mat/0106152}

\sbibitem{2003a}
{\LF, A.J. Niemi, and W. Ulrich}
{Glueballs, closed fluxtubes, and eta(1440)}
{Phys.\ Rev.} {D70} {} {114033} {2004}
\arx{hep--ph/0308240}

\rbibitem{2004a}
{Т.А. Болохов и \LD}
{Инфракрасные переменные для SU(3) поля Янга--Миллса}
{\TMF} {139} {no.~2} {276--290} {2004}

\ebibitem{2004a}
{T.A. Bolokhov and \LF}
{Infrared variables for the SU(3) Yang--Mills field}
{Theor.\ Math.\ Phys.} {139} {679--692} {2004}

\bibiteme1{2005b}
{\LF}
{What is complete integrability in quantum mechanics}
{\em Proc. of the Symposium H.~Poincar\'e}
{Solvay Institute} {Brussels} {2004} {} {9 pp.} {2} {}

\bibitemrp{2005b}
{en}
{\em Nonlinear equations and spectral theory (AMS Transl. Ser. 2)}
{AMS} {Providence, RI} {2007} {220} {83--90} {1}

% Solvay Conference, dedicated to the 150 anniversary of

\bibiteme1{2004b}
{\LF} 
{Algebraic lessons from the theory of quantum integrable models} 
{\em The unity of mathematics: In honor of the ninetieth 
birthday of I.M. Gelfand. Progress in Mathematics. (Int. conf. MIT, 2003)} 
{Birkh\"auser} {} {2006} {224} {305--320} {2} {}

\sbibitem{2006a}
{\LF}
{History and perspectives of quantum groups
 (Leonardo da Vinci lecture, 2005)}
{Milan Journal of Mathematics} {74} {} {279--294} {2006}

\rbibitem{2006b}
{\LD}
{Замечания о расходимостях и размерной трансмутации в теории Янга–-Миллса}
{\TMF} {148} {no.~1} {133--142} {2006}

\ebibitem{2006b}
{\LF}
{Notes on divergences and dimensional transmutation
in Yang--Mills theory}
{Theor. Math. Phys.} {148}  {986--994} {2006}

\bibiteme1{2006d}
{\LF}
{Discretized Virasoro algebra}
{\em Noncommutative geometry and representation theory in
mathematical physics. Contemporary Mathematics} 
{AMS} {Providence, RI} {2006} {391} {59--67} {3} {}

\sbibitem{2006c}
{\LF\ and A.J. Niemi}
{Spin--charge separation, conformal covariance, and
the SU(2) Yang-Mills theory}
{Nuclear Phys.} {B776} {} {38--65} {2007}
\arx{hep--th/0608111}

\rbibitem{2007a}
{\LD}
{Дискретная серия представлений модулярного
дубля $U_q(sl(2,{\mathbb R}))$}
{\FA} {42} {no.~4} {98--104} {2008}

\ebibitem{2007a}
{\LF}
{Discrete series of representations for the modular
double of $U_q(sl(2,{\mathbb R}))$}
{Funct. Anal. Appl.} {42} {330--335} {2008}
\\ arXiv: 0712.2747 [math.QA]


\sbibitem{2008a}
{\LF\ and A.Yu. Volkov}
{ Discrete evolution for the zero--modes of the quantum Liouville model}
{J. Phys. A: Math. Theor.} {41} {} {194003} {2008}
\\ arXiv: 0803.0230 [hep-th]

\sbibitem{2008b}
{M.N. Chernodub, \LF, and A.J. Niemi}
{ Non--abelian supercurrents and de Sitter ground state in electroweak theory}
{Journal of High Energy Physics}{12}{}{014} {2008}
\\ arXiv: 0804.1544 [hep-th]


\bibiteme1{2008c}
{\LF}
{Knots as possible excitations of the quantum Yang--Mills fields}  
{Quantum field theory and beyond: Essays in honor 
of Wolfhart Zimmermann} 
{World Scientific} {} {2008} {} {156--166} {3}
{0805.1624 [hep-th]}
% E.Seiler, K.Sibold (eds.)
  

\bibiteme1{2009a}
{\LF}
{An alternative interpretation of the Weinberg-Salam model}
{Progress in high energy physics and nuclear safety.
(NATO Science for Peace and Security Series B: 
Physics and Biophysics} {} {} {2009} {} {3--8}
{0811.3311 [hep-th]}


\bibiteme1{2009b}
{\LF}
{New action for the Hilbert--Einstein equations}
{arxiv} {} {} {2009} {} {1-6} {1}
{0906.4639}

\sbibitem{2010a}
{\LF}
{Faddeev--Popov ghosts}
{Int. J. Mod. Phys. A} {25} {} {1079--1089} {2010}

\bibiteme1{2010b}
{\LF}
{Separation of scattering and selfaction revisited}
{\em Subtleties in quantum field theory (Lev Lipatov Festschrift, ed. D. Diakonov)} {Petersburg Nuclear Physics Institute} {Gatchina} {2010} {} {1--6} {2} {}

\bibiteme1{2010c}
{\LF}
{3+1 decomposition in the new action for the Einstein theory of gravitation}
{arxiv} {} {} {2010} {} {1-9} {1} {1003.2311}

\rbibitem{2011b}
{\LD} 
{Новые динамические переменные теории тяготения Эйнштейна} 
{\TMF} {166} {no.~3} {323--335} {2011} 

\ebibitem{2011b}
{\LF}
{New dynamical variables in Einstein’s theory of gravity}
{Theor. Math. Phys.} {166} {279--290} {2011}
arXiv: 0911.0282 [hep-th]


\rbibitem{2011a}
{\LD} 
{Пентагон Волкова для модулярного квантового дилогарифма}
{\FA} {45} {no.~4} {65--71} {2011}

\ebibitem{2011a}
{\LF}
{Volkov pentagon for the modular quantum dilogarithm}
{Funct. Anal. Appl.} {45} {291--296} {2011}
arXiv: 1201.6464 [math.QA]

\bibiteme1{2013a}
{S.E. Derkachov and L.D. Faddeev}
{3j-symbol for the modular double of $SL_q(2,{\mathbb R})$ revisited}
{arxiv} {} {} {2013} {} {1-16} {1} {1302.5400}

\rbibitem{2013b}
{\LD}
{Новая жизнь полной интегрируемости}
{\UFN} {183} {no.~5} {487--495} {2013} 

\sbibitem{2013b}
{\LF}
{The new life of complete integrability} 
{Physics--Uspekhi} {56} {n.5} {465} {2013}

\rbibitem{2013c}
{\LD}
{Примеры гамильтоновых структур в 
теории интегрируемых моделей и их квантование}
{Алгебра и анализ} {25} {no.~2} {193--202} {2013}


